\section{Communication Serveur/GUI}

\subsection{Présentation}

\par La communication entre le serveur de l'application et les conteneurs repose sur une connexion SSH. L'application se connecte au serveur contenant les dockers afin de communiquer les messages et d'exécuter les commandes qui lui sont donnés.

\subsection{Outils utilisés}

\par L'utilisation de Symfony a permis l'utilisation de la librairie libssh2 de PHP. Cette librairie permet de gérer facilement une connexion SSH en PHP.

\subsection{Processus de communication}

\par Le processus de communication repose sur un système de questions/réponses. Le client demande des informations et le serveur lui répond.

\subsubsection{Envoi des requêtes et réception des réponses}

\paragraph*{Première requête envoyée (appui sur le bouton Lancer)} \

\par L'application reçoit la requête du client. Elle va créer la commande qui permet de démarrer un conteneur docker et de lancer le script qui se charge de la compilation et de l'exécution du programme. La commande est ensuite passé au service GestionSSH qui se charge de se connecter en SSH au serveur qui contient les conteneurs et d'exécuter la commande grâce à un shell. 

\par L'application récupère ensuite la sortie standard du programme, toujours grâce au service GestionSSH. Elle répond ensuite au client, qui attend toujours la réponse du serveur. Le client affiche ensuite la réponse dans la vue jqconsole.

\paragraph*{Envoi des requêtes suivantes} \

\par Dans certains programmes, l'utilisateur a besoin de répondre au programme pour que celui-ci se termine (fonction d'inputs). Dans ce cas, l'application teste si le docker est encore ouvert et le redémarre si c'est le cas. Le message est ensuite passé au service de GestionSSH qui se charge de la transmettre au conteneur grâce à SSH.

\par L'application récupère de nouveau la sortie standard du programme pour l'envoyer au client. Cette opération se répète jusqu'à ce que l'exécution du programme soit terminée.

\subsection{Architecture des classes}