\section{Communication Serveur/GUI}

\subsection{Outils utilisés}

libssh2, service symfony
La communication entre le front end et le back end contient deux classes principales : ConsoleController et GestionSSH.

\subsection{Processus de communication}

\subsubsection{Envoi des requêtes}

\par L’utilisateur saisie son programme dans l’éditeur de texte et appuie sur le bouton Run. Une requête est ensuite envoyée à l'application qui va récupérer le ou les fichiers à compiler, ouvrir une connexion SSH avec le serveur contenant les images docker. Un conteneur est ensuite crée afin de lancer la compilation et l’exécution du programme grâce à un script. 

\subsubsection{Réception des réponses du serveur}

\par Une fois l'exécution lancée, l'application lis la sortie standard du conteneur. Le texte récupéré est ensuite retourné à l'interface utilisateur qui va l'afficher dans la vue jqconsole.

\par Dans les cas où le programme nécessite des entrées, l'utilisateur entrera ses messages dans la vue jqconsole. Ces messages seront ensuite envoyés à l'application qui se chargera de rouvrir une connexion SSH avec le serveur et d'envoyer les messages au conteneur.