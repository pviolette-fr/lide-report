%Paulin VIOLETTE
%Si t'as pas bossé sur l'interface utilisateur t'y touche pas.
%Sauf si la personne dont le nom est écris en haut te dis d'y toucher.
\chapter{Interface Utilisateur}

\section{Présentation}

%Mettre l'image de ma super interface grave stylé
\begin{figure}[h]
  \centering
  \includegraphics[width=0.8\textwidth]{./frontend/example1.png}
  \caption{Interface utilisateur en utilisation}
  \label{}
\end{figure}

Une fois l'utilisateur connecté, il est redirgigé vers l'interface de l'application : un éditeur de texte et une console.
L'interface est divisée en quatre parties :
\begin{itemize}
  \item La barre de navigation, contenant les liens vers les autres parties du site (gestion de compte...)
  \item La barre d'outils, qui contient des contrôles spécifique à l'application.
  \item L'editeur, implémenté par le plugin Ace
  \item La console, implémenté par le plugin jqconsole.
\end{itemize}

\section{Outils utilisés}

%Blabla HTML/CSS/JS, generateur de template twig, Framework bootstrap
%Utilisation de jquery
%Editeur ace
%SweetAlert2 pour les alertes trop swag
%JQConsole vite zef parceque c'est valou qui l'a fait

\subsection{Organisation des templates TWIG}

\section{Environnement de Développement}

\subsection{Gestions des langages}
Blabla DB changement de langage

\subsection{Éditeur de texte}
Tres court parce que y'a pas grand chose à dire

\subsection{Personnalisation}

Toute personne ayant travaillé en groupe sur un projet informatique à pu remarquer que chacun à ces préference de thème pour un éditeur : certains préfere un fond sombre, d'autre un fond clair, etc...
L'éditeur Ace est facilement personnalisable, et dispose par défault de 24 thèmes. Il était donc assez rapide d'implementer à formulaire permettant à l'utilisateur de choisir le thème qui lui convient le mieux,
lui permettant ainsi de facilement s'approprier son outil de travail.

La police est également personnalisable, permettant à chacun d'utiliser une taille de police qui lui convient.

La console ayant un style implementer par le css, il fut de même aisé de créer des thèmes qui s'applique grâce à une classe attribué à l'élement div contenant la console.
Pour l'instant, seul trois style de console sont implémenter, mais il serait aisé d'en ajouté d'autre dans des versions future.
Chaque style a une classe maitresse \emph{.console-nom\_style}, et on redéfini ensuite les classes \emph{jqconsole} grâce aux selecteurs css.

\begin{figure}[!h]
\centering
\includegraphics[width=0.8\textwidth]{./frontend/example_personnalisation.png}
\caption{Formulaire permettant la personnalisation de l'interface}
\end{figure}

\subsection{Gestions des fichiers}
Tel un véritable EDI, notre application permet la gestion de multiples fichiers. Cette gestion est effectué sur le navigateur, par du javascript.

Les fichiers sont enregistré dans un prototype contenant deux champs : \emph{name}, le nom du fichier, et \emph{content} son contenu. Ces prototype sont ensuite stockés dans la variable
globale \emph{files}, un tableau de fichier. À chaque changement de fichier en cours d'édition, le contenu de l'éditeur est sauvegardé et ensuite remplacé par le contenu du nouveau fichier à éditer.

Les fichiers peuvent être créer de deux façon : soit en important un fichier depuis sont ordinateur (bouton importation), soit par création à partir de modèles défini par l'administrateur pour le langage.
On laisse également la possibilté de créer des fichiers vides (par exemple pour créer un fichier de données utilisé lors de l'éxecution du programme).


\section{Compilation et exécution}
Ce titre pu

\subsection{Formulaire}

SS de mon formulaire trop swag

\subsection{Terminal}

Vas y valou tu 'écrira ça
