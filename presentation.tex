\chapter{Présentation du projet}

\par L'installation de compilateurs peut se montrer compliquée pour les étudiants novices qui suivent les cours d'initiation à la programmation. L'Université d'Angers a donc souhaité simplifier leur apprentissage en centralisant tous les besoins dans un outil de développement en ligne.

\section{Sujet}
%Présentation du sujet : entreprise, encadrement

\par Notre objectif était de réaliser la première version d'un environnement de développement simplifié accessible à partir d'un navigateur web. Il permet d'écrire et de compiler du code rapidement et simplement sans avoir à installer de compilateur sur le poste client (la compilation s'effectuant sur le serveur). Cet outil pourrait être utilisé, dans le cadre des enseignements à l'Université d'Angers, dans plusieurs unités dès la L1 jusqu'à la L3. \\

\par Certaines caractéristiques nous étaient demandées : 

\begin{itemize}

	\item l'édition de code dans un éditeur proposant la coloration syntaxique
	\item la possibilité de compiler le code depuis des compilateurs installés sur le serveur. Le logiciel devra prendre en compte différents langages afin d’être utilisable dans différentes UE
	\item l'accès aux messages d'erreur de la compilation
	\item l'exécution de l'application compilée sur le serveur avec affichage de la sortie
	\item des fonctionnalités avancées devaient être développées telles que l’intégration du débogueur, une gestion plus poussée de l’ensemble de fichiers composant un projet, l'auto-complétion du code, l'intégration d’outils d’analyse

\end{itemize}


\section{Problématique soulevée}

\par L'application devant être intégrée aux serveurs de l'Université, nous nous devions de penser à la sécurité. Nous avons donc élaborer une architecture sécurisée afin d'éviter les failles. 

\section{Choix des principaux outils et technologies}

Blabal symfony mariaDB Bootstrap Docker blablabla

\section{Répartition des tâches}

\par Afin de travailler efficacement, nous avons séparé le projet en 4 parties autonomes plus ou moins autonomes: Paulin s'est occupé de l'interface graphique, Yassine de l'administration, Jérôme de l'architecture du serveur et Valentine de la communication entre les serveurs.

\subsection{Planning}

Faire le planning