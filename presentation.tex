\chapter{Présentation du projet}

Intro\footnotemark\\
%note en bas de page

\par L'installation de compilateurs peut se montrer compliquée pour les étudiants novices qui suivent les cours d'initiation à la programmation. L'Université d'Angers a donc souhaité simplifier leur apprentissage en centralisant tous les besoins dans un outil de développement en ligne.

\section{Sujet}
%Présentation du sujet : entreprise, encadrement
Bla\\

\par Notre objectif était de réaliser un environnement de développement en ligne. Cet outil , qui serait utilisé essentiellement par les étudiants en première et deuxième année de licence à l'Université d'Angers, a pour intérêt principal de permettre l’écriture, la compilation et l’exécution de programme sans installation de compilateurs ou d'IDE préalable.

\section{Problématique soulevée}

Sécurité au niveau du serveur

\section{Choix des principaux outils et technologies}

Blabal symfony mariaDB Bootstrap Docker blablabla

\section{Répartition des tâches}

\par Afin de travailler efficacement, nous avons séparé le projet en 4 parties autonomes : l'interface graphique, l'administration, le serveur et la communication entre serveur et interface graphique.