\chapter{Présentation du projet}

\par L'installation de compilateurs peut se montrer compliquée pour les étudiants novices qui suivent les cours d'initiation à la programmation. L'Université d'Angers a donc souhaité simplifier leur apprentissage en centralisant tous les besoins dans un outil de développement en ligne.

\section{Sujet}
%Présentation du sujet : entreprise, encadrement

\par Notre objectif était de réaliser la première version d'un environnement de développement simplifié accessible à partir d'un navigateur web. Il permet d'écrire et de compiler du code rapidement et simplement sans avoir à installer de compilateur sur le poste client (la compilation s'effectuant sur le serveur). Cet outil pourrait être utilisé, dans le cadre des enseignements à l'Université d'Angers, dans plusieurs unités dès la L1 jusqu'à la L3. \\

\par Certaines caractéristiques nous étaient demandées : 

\begin{itemize}

	\item l'édition de code dans un éditeur proposant la coloration syntaxique
	\item la possibilité de compiler le code depuis des compilateurs installés sur le serveur. Le logiciel devra prendre en compte différents langages afin d’être utilisable dans différentes UE
	\item l'accès aux messages d'erreur de la compilation
	\item l'exécution de l'application compilée sur le serveur avec affichage de la sortie
	\item des fonctionnalités avancées devaient être développées telles que l’intégration du débogueur, une gestion plus poussée de l’ensemble de fichiers composant un projet, l'auto-complétion du code, l'intégration d’outils d’analyse

\end{itemize}


\section{Problématique soulevée}

\par L'enjeu principal de ce projet était la sécurité du serveur. En effet, lorsqu'un étudiant tente d'exécuter un programme qui contient des erreurs ou qui demande trop de mémoire CPU, il ne faut pas que le serveur qui gère l'exécution ni les exécutions d'autres étudiants soient ralentis ou bloqués. Pour pallier ces problèmes, nous nous devions d'élaborer une architecture sécurisée afin d'éviter les failles.


\section{Choix des principaux outils et technologies}

\par Lors de la première séance de concrétisation disciplinaire, nos chefs de projet nous ont proposé d'utiliser le framework Symfony pour la réalisation de notre application. Ce framework force à organiser le code et permet une gestion simple de la base de données qui ne dépend pas du type de base de données. De plus, Symfony permet une génération simple des pages grâce à ses controllers et au moteur de template twig.\
Un autre avantage de Symfony est la possibilité d'utiliser des bundles (par exemple, FOSUserBundle permet la gestion des utilisateurs) qui simplifie et accélère véritablement la réalisation des projets. \\

\par Nous avons décidé de placer sous licence libre notre application, c'est pourquoi nous avons choisi MariaDB comme système de gestion de la base de données. De plus, MariaDB a l'avantage d'être légère. \\

\par L'interface graphique a été réalisé grâce au framework Bootstrap qui fournit un thème de base cohérent et esthétique. Bootstrap est simple à utiliser car il se base sur des classes CSS. \\

\par Le choix de la conteneurisation, et plus particulièrement Docker, s'est vite imposé dans l'architecture du serveur. Docker est une méthode de conteneurisation légère qui permet de travailler toujours sur le même environnement (la même image est réutilisée autant de fois que nécessaire) et qui nous a permis d'isoler les compilations et exécutions des programmes.

\section{Répartition des tâches}

\par Afin de travailler efficacement, nous avons séparé le projet en 4 parties autonomes plus ou moins autonomes: Paulin s'est occupé de l'interface graphique, Yassine de l'administration, Jérôme de l'architecture du serveur et Valentine de la communication entre les serveurs.

\subsection{Planning}

Faire le planning